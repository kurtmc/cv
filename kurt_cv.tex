\documentclass[11pt,a4paper,sans]{moderncv}        % possible options include font size ('10pt', '11pt' and '12pt'), paper size ('a4paper', 'letterpaper', 'a5paper', 'legalpaper', 'executivepaper' and 'landscape') and font family ('sans' and 'roman')

% moderncv themes
\moderncvstyle{casual}                             % style options are 'casual' (default), 'classic', 'oldstyle' and 'banking'
\moderncvcolor{red}                               % color options 'blue' (default), 'orange', 'green', 'red', 'purple', 'grey' and 'black'

% adjust the page margins
\usepackage[scale=0.75]{geometry}
%\setlength{\hintscolumnwidth}{3cm}                % if you want to change the width of the column with the dates
%\setlength{\makecvtitlenamewidth}{10cm}           % for the 'classic' style, if you want to force the width allocated to your name and avoid line breaks. be careful though, the length is normally calculated to avoid any overlap with your personal info; use this at your own typographical risks...

% personal data
\name{Kurt}{McAlpine}
%\title{Linux Developer}                               % optional, remove / comment the line if not wanted
\address{3D/18 Federal Street}{1010 Auckland}{New Zealand}% optional, remove / comment the line if not wanted; the "postcode city" and "country" arguments can be omitted or provided empty
\phone[mobile]{+64~220~459~539}                   % optional, remove / comment the line if not wanted; the optional "type" of the phone can be "mobile" (default), "fixed" or "fax"
%\phone[fixed]{+64~942~622~18}
%\phone[fax]{+3~(456)~789~012}
\email{kurt@linux.com}                               % optional, remove / comment the line if not wanted
%\homepage{www.kurt-mcalpine.com}                         % optional, remove / comment the line if not wanted
%\social[linkedin]{john.doe}                        % optional, remove / comment the line if not wanted
%\social[twitter]{jdoe}                             % optional, remove / comment the line if not wanted
\social[github]{kurtmc}                              % optional, remove / comment the line if not wanted
%\extrainfo{additional information}                 % optional, remove / comment the line if not wanted
\photo[64pt][0.4pt]{picture}                       % optional, remove / comment the line if not wanted; '64pt' is the height the picture must be resized to, 0.4pt is the thickness of the frame around it (put it to 0pt for no frame) and 'picture' is the name of the picture file

% These are defined so that I can use tabular
\setlength\arrayrulewidth{.4pt}
\setlength\tabcolsep{6pt}

\begin{document}

\makecvtitle

\section{Education}
\cventry{2012--2015}{B.Eng.(Hons)--Software Engineering}{University of Auckland}{Auckland}{}{}  % arguments 3 to 6 can be left empty

\section{Experience}
\cventry{2016--present}{Developer}{EROAD}{Auckland}{}{
	In 2016 I spent 3 months working the the operations team. I was responsible for:
	\begin{itemize}
		\item deploying particular versions of software to the staging environments
		\item production deploys using Ansible
		\item automating the configuration of long lived servers using Chef
	\end{itemize}
	Currently work in the Architecture team. I am responsible for:
	\begin{itemize}
		\item Migrating old style deploys from Chef and Ansible to
			\href{https://concourse.ci/}{Concourse CI} pipelines
			that build, test and deploy software once the pull
			request is merged
		\item Using AWS CloudFormation to build stacks that includes EC2
			auto scaling groups, RDS, Kinesis Streams, Redis Caches,
			ELB's
		\item Using the AWS API to automate the cleanup of AWS resources
			such as AMI's and snapshots
		\item Using the AWS API to create a large pool of EC2 EBS
			volumes with identical data, for the purpose of having a
			massively distributed read only database.
		\item Automating the deployment of Chef resources such as
			environment state and cookbooks
	\end{itemize}
}
\cventry{2015--2016}{Software Engineering Consultant}{Alchemy Agencies}{Auckland}{}{
Worked as a consultant to make improvements to existing infrastructure:
Activities performed:
\begin{itemize}
\item Creation of simple CMS to manage products and documents associated with
	those products using Ruby on Rails
\item Integrating the Ruby on Rails application with existing WordPress
	application allowing users on the WordPress application to login and
	download documents associated with products they have been
	granted to read
\end{itemize}
}
\cventry{2014--2015}{Android Developer}{Vista Entertainment Solutions}{Auckland}{}{Worked as an Android developer on a consumer application that allows customers
to buy tickets to see movies at their local cinema.\\
My work on the Android application involves:
\begin{itemize}
\item collaborating with other developers, test analysts, business analysts and business people
\item using a bug tracker to find bugs relating to the application
\item using version control systems to get the latest source code from a centralised server, merging
changes into the main and release branches
\item fixing internal and customer visible bugs
\item implementing features designed by the business analyst on the team
\item having my work reviewed and quality tested
\end{itemize}}
\cventry{2013--2014}{Student Intern}{Vista Entertainment Solutions}{Auckland}{}{
\begin{itemize}%
\item My role involved working on bugs in the cinema management software. The technologies used in that
area include WinForms, Visual Basic, C\#, .NET, and Microsoft SQL Server.
\item Some of my work included web development where I updated the user interface on a legacy website
that served pages using XSLT. Other technologies used were JavaScript, SASS, HTML, and ASP.NET.
\end{itemize}}

\section{Personal Projects}
	\cventry{Go}{cloudformation-resource}{\url{https://github.com/ci-pipeline/concourse-ci-resource}}{}{}{
		This is a Concourse CI resource that allows you to deploy
		CloudFormation stacks to AWS. It issues create or update
		commands and logs the cloudformation events to the console.
		}
	\cventry{Go}{packer-resource}{\url{https://github.com/ci-pipeline/packer-resource}}{}{}{
		This is a Concourse CI resource that allows you to build AWS
		AMI's and Docker containers based on packer scripts. It outputs
		the AWS AMI ID as a file which can be used elsewhere in the
		Concourse pipeline.
		}
	\cventry{Go}{ec2search}{\url{https://github.com/kurtmc/ec2search}}{}{}{
		Command line tool and library to search for EC2
		instances by name in all regions.
		}
	\cventry{Go}{ec2ssh}{\url{https://github.com/kurtmc/ec2ssh}}{}{}{
		Command line tool to ssh into a random EC2 instance that
		matches the search criteria or run a command on all
		instance that match the search criteria.
		}
	\cventry{Go}{prepare-commit-msg}{https://github.com/kurtmc/prepare-commit-msg}{}{}{
		Git hook to prepend the current branch name to commit
		messages. This is used to link commits to Jira tickets
		}

\section{References available upon request}

\clearpage

%-----       letter       ---------------------------------------------------------

% Bunch of commands so that I can swap out different company names and people
\newcommand{\companyName}{<company name>}
\newcommand{\companyAddress}{Auckland\\New Zealand}
\newcommand{\roleName}{<role name>}
\newcommand{\recipientName}{Sir / Madam}

% recipient data
\recipient{\companyName{}}{\companyAddress{}}
\date{\today}
\opening{Dear \recipientName{},}
\closing{Yours faithfully,}
\enclosure[Attached]{curriculum vit\ae{}}
\makelettertitle

This application is for \roleName{} advertised by \companyName{}.

I have completed a four year Software Engineering degree at the University of
Auckland. I think that I can be extremely valuable to your company as I have a
great deal of real world experience as well as contemporary knowledge gained
through my university education.

I looking forward to meeting with you in the future to discuss this further.

\makeletterclosing

\end{document}
