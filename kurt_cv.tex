\documentclass[11pt,a4paper,sans]{moderncv}        % possible options include font size ('10pt', '11pt' and '12pt'), paper size ('a4paper', 'letterpaper', 'a5paper', 'legalpaper', 'executivepaper' and 'landscape') and font family ('sans' and 'roman')

% moderncv themes
\moderncvstyle{casual}                             % style options are 'casual' (default), 'classic', 'oldstyle' and 'banking'
\moderncvcolor{grey}                               % color options 'blue' (default), 'orange', 'green', 'red', 'purple', 'grey' and 'black'

% adjust the page margins
\usepackage[scale=0.85]{geometry}
%\setlength{\hintscolumnwidth}{3cm}                % if you want to change the width of the column with the dates
%\setlength{\makecvtitlenamewidth}{10cm}           % for the 'classic' style, if you want to force the width allocated to your name and avoid line breaks. be careful though, the length is normally calculated to avoid any overlap with your personal info; use this at your own typographical risks...

% personal data
\name{Kurt}{McAlpine}
%\title{Linux Developer}                               % optional, remove / comment the line if not wanted
\address{1/9 New Windsor Road, Avondale}{0600 Auckland}{New Zealand}% optional, remove / comment the line if not wanted; the "postcode city" and "country" arguments can be omitted or provided empty
\phone[mobile]{+64~220~459~539}                   % optional, remove / comment the line if not wanted; the optional "type" of the phone can be "mobile" (default), "fixed" or "fax"
%\phone[fixed]{+64~942~622~18}
%\phone[fax]{+3~(456)~789~012}
\email{kurtmcalpine@gmail.com}                               % optional, remove / comment the line if not wanted
%\homepage{www.kurt-mcalpine.com}                         % optional, remove / comment the line if not wanted
\social[linkedin][www.linkedin.com/in/kurt-mcalpine]{kurt.mcalpine}                        % optional, remove / comment the line if not wanted
%\social[twitter]{jdoe}                             % optional, remove / comment the line if not wanted
\social[github]{kurtmc}                              % optional, remove / comment the line if not wanted
%\extrainfo{additional information}                 % optional, remove / comment the line if not wanted
%\photo[64pt][0.4pt]{picture}                       % optional, remove / comment the line if not wanted; '64pt' is the height the picture must be resized to, 0.4pt is the thickness of the frame around it (put it to 0pt for no frame) and 'picture' is the name of the picture file

% These are defined so that I can use tabular
\setlength\arrayrulewidth{.4pt}
\setlength\tabcolsep{6pt}

\begin{document}

\makecvtitle

\section{Experience}
\cventry{2021--present}{Consultant}{Sourced Group}{Auckland}{}{
	\begin{itemize}
        \item Deliver internal talks to uplift team skills and knowledge
        \item With with other consultants on internal projects that deliver
            value to the business internally
        \item Work with enterprise clients to accelerate their cloud journey
	\end{itemize}
}
\cventry{2020--2021}{Lead Cloud Engineer}{Halter}{Auckland}{}{
	\begin{itemize}
        \item Conduct technical interviews for potential hires
        \item Up skill new employees on systems and processes during on boarding
        \item Train other departments on internal tooling to enhance
            capabilities within the company
        \item Point of escalation for cloud infrastructure and customer
            networking issues and design decisions
	\end{itemize}
}
\cventry{2018--2020}{DevOps Engineer}{Halter}{Auckland}{}{
	\begin{itemize}
		\item Deploy multi account AWS (Amazon Web Services) structure using IaC (Infrastructure as Code) and automatic deployments
		\item Establish CI/CD (Continuous Integration / Continuous Delivery) patterns for deploying applications and core infrastructure
		\item Implement CI/CD process for building, testing and
			deploying of:
			\begin{itemize}
				\item backend applications
				\item mobile applications using Concourse CI, AWS Device Farm and Fastlane
				\item embedded firmware using Concourse CI and AWS IoT
			\end{itemize}
        \item Optimise developer workflow by building CLI tool to allow
            developers to iterate on IaC quickly and safely
	\end{itemize}
}
\cventry{2017--2018}{DevOps Team Lead}{Movio}{Auckland}{}{
	\begin{itemize}
		\item Promote the use of IaC
		\item Adopted CI/CD strategy to deploy applications and infrastructure
		\item Reduce manual processes for on boarding new customers by
			way of automation thus reducing human error and
			speeding up on boarding process
		\item Begin migration from hand rolled database deployments to
			AWS RDS for increased easy of use, automated backups,
			automated failover and increased visibility through
			monitoring.
	\end{itemize}
}
\cventry{2016--2017}{Developer}{EROAD}{Auckland}{}{
	\begin{itemize}
		\item Automating the configuration of long lived servers using Chef
		\item Migrating old style deploys from Chef and Ansible to
			\href{https://concourse.ci/}{Concourse CI} pipelines
			that build, test and deploy software once the pull
			request is merged
		\item Using AWS CloudFormation to build stacks that includes EC2
			auto scaling groups, RDS, Kinesis Streams, Redis Caches,
			ELB's
        \item Automate the clean up of AWS resources such as AMI's and
            snapshots
        \item Using the AWS API to create a large pool of EC2 EBS volumes with
            identical data, for the purpose of having a massively distributed
            read only database
		\item Automating the deployment of Chef resources such as
			environment state and cookbooks
	\end{itemize}
}
%\cventry{2015--2016}{Software Engineering Consultant}{Alchemy Agencies}{Auckland}{}{
%Worked as a consultant to make improvements to existing infrastructure.
%Key accomplishments:
%\begin{itemize}
%	\item Created a simple CMS to manage products and documents associated
%		with those products using Ruby on Rails
%	\item Integrated the Ruby on Rails application with existing WordPress
%		application allowing users on the WordPress application to login
%		and download documents associated with products they have been
%		granted to read
%\end{itemize}
%}
%\cventry{2014--2015}{Android Developer}{Vista Entertainment Solutions}{Auckland}{}{Worked as an Android developer on a consumer application that allows customers
%to buy tickets to see movies at their local cinema.\\
%Work on the Android application involved:
%	\begin{itemize}
%		\item Collaborating with other developers, test analysts, business analysts and business people
%		\item Using a bug tracker to find bugs relating to the application
%		\item Using version control systems to get the latest source
%			code from a centralised server, merging changes into the
%			main and release branches
%		\item Fixing internal and customer visible bugs
%		\item Implementing features designed by the business analyst on the team
%		\item Having my work reviewed and quality tested
%	\end{itemize}
%}
%
%\cventry{2013--2014}{Student Intern}{Vista Entertainment Solutions}{Auckland}{}{
%	\begin{itemize}%
%		\item Working on bugs in the cinema management software. The
%			technologies used in that area include WinForms, Visual
%			Basic, C\#, .NET, and Microsoft SQL Server.
%		\item Web development, updating the user interface on a legacy
%			website that served pages using XSLT. Other technologies
%			used were JavaScript, SASS, HTML, and ASP.NET.
%	\end{itemize}
%	}

\section{Education}
\cventry{2012--2015}{B.Eng. (Hons)--Software Engineering}{University of Auckland}{Auckland}{}{}  % arguments 3 to 6 can be left empty

\section{Open Source and Personal Projects}
	\cventry{Go}{terraform-provider-aws}{\tiny{\url{https://github.com/hashicorp/terraform-provider-aws}}}{}{}{
        Contributions to the AWS Terraform provider, including:
        \begin{itemize}
            \item \href{https://github.com/hashicorp/terraform-provider-aws/pull/25109}{Add new aws\_acmpca\_policy resource}
            \item \href{https://github.com/hashicorp/terraform-provider-aws/pull/14905}{Add cidrs attribute to aws\_lightsail\_instance\_public\_ports}
            \item \href{https://github.com/hashicorp/terraform-provider-aws/pull/11922}{Implement error\_action on aws\_iot\_topic\_rule}
        \end{itemize}
        }
	\cventry{Go}{terraform-provider-plausible}{\tiny{\url{https://github.com/mcalpinefree/terraform-provider-plausible}}}{}{}{
        Plugin for \href{https://www.terraform.io/}{Terraform} that allows the
        creation and modification of resources in in
        \href{https://plausible.io/}{Plausible Analytics} using Terraform.
		}
	\cventry{Go}{terraform-provider-unifi}{\tiny{\url{https://github.com/paultyng/terraform-provider-unifi}}}{}{}{
        Major contributions to Terraform provider for Unifi. This provider
        allows the configuration of networks on Unifi Network Controller.
		}

\section{References available upon request}

\clearpage

%-----       letter       ---------------------------------------------------------

% Bunch of commands so that I can swap out different company names and people
\newcommand{\companyName}{<company name>}
\newcommand{\companyAddress}{Auckland\\New Zealand}
\newcommand{\roleName}{<role name>}
\newcommand{\recipientName}{Sir / Madam}

% recipient data
\recipient{\companyName{}}{\companyAddress{}}
\date{\today}
\opening{Dear \recipientName{},}
\closing{Yours faithfully,}
\enclosure[Attached]{curriculum vit\ae{}}
\makelettertitle

This application is for \roleName{} advertised by \companyName{}.

I have completed a four year Software Engineering degree at the University of
Auckland. I think that I can be extremely valuable to your company as I have a
great deal of real world experience as well as contemporary knowledge gained
through my university education.

I looking forward to meeting with you in the future to discuss this further.

\makeletterclosing

\end{document}
