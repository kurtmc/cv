\documentclass[11pt,a4paper,sans]{moderncv}        % possible options include font size ('10pt', '11pt' and '12pt'), paper size ('a4paper', 'letterpaper', 'a5paper', 'legalpaper', 'executivepaper' and 'landscape') and font family ('sans' and 'roman')

% moderncv themes
\moderncvstyle{casual}                             % style options are 'casual' (default), 'classic', 'oldstyle' and 'banking'
\moderncvcolor{red}                               % color options 'blue' (default), 'orange', 'green', 'red', 'purple', 'grey' and 'black'

% adjust the page margins
\usepackage[scale=0.75]{geometry}
%\setlength{\hintscolumnwidth}{3cm}                % if you want to change the width of the column with the dates
%\setlength{\makecvtitlenamewidth}{10cm}           % for the 'classic' style, if you want to force the width allocated to your name and avoid line breaks. be careful though, the length is normally calculated to avoid any overlap with your personal info; use this at your own typographical risks...

% personal data
\name{Kurt}{McAlpine}
%\title{Linux Developer}                               % optional, remove / comment the line if not wanted
\address{3D/18 Federal Street}{1010 Auckland}{New Zealand}% optional, remove / comment the line if not wanted; the "postcode city" and "country" arguments can be omitted or provided empty
\phone[mobile]{+64~220~459~539}                   % optional, remove / comment the line if not wanted; the optional "type" of the phone can be "mobile" (default), "fixed" or "fax"
%\phone[fixed]{+64~942~622~18}
%\phone[fax]{+3~(456)~789~012}
\email{kurt@linux.com}                               % optional, remove / comment the line if not wanted
\homepage{www.kurt-mcalpine.com}                         % optional, remove / comment the line if not wanted
%\social[linkedin]{john.doe}                        % optional, remove / comment the line if not wanted
%\social[twitter]{jdoe}                             % optional, remove / comment the line if not wanted
\social[github]{kurtmc}                              % optional, remove / comment the line if not wanted
%\extrainfo{additional information}                 % optional, remove / comment the line if not wanted
\photo[64pt][0.4pt]{picture}                       % optional, remove / comment the line if not wanted; '64pt' is the height the picture must be resized to, 0.4pt is the thickness of the frame around it (put it to 0pt for no frame) and 'picture' is the name of the picture file

% These are defined so that I can use tabular
\setlength\arrayrulewidth{.4pt}
\setlength\tabcolsep{6pt}

\begin{document}

\makecvtitle

\section{Education}
\cventry{2012--2015}{B.Eng.(Hons)--Software Engineering}{University of Auckland}{Auckland}{}{}  % arguments 3 to 6 can be left empty
%\cvitem{Current GPA}{6.4}
\subsection{Key Grades}
\cvitem{A-}{SOFTENG 250 - Introduction to Data Structures}
\cvitem{A-}{SOFTENG 211 - Software Engineering Theory}
\cvitem{A-}{SOFTENG 370 - Operating Systems}
\cvitem{A-}{SOFTENG 351 - Fundamentals of Database Systems}

\section{Experience}
\subsection{Vocational}
\cventry{2014--2015}{Android Developer}{Vista Entertainment Solutions}{Auckland}{}{Currently working as a part time Android developer on a consumer application that allows customers
to buy tickets to see movies at their local cinema.\\
Link demo application: \url{https://play.google.com/store/apps/details?id=nz.co.vista.android.movie.abc}\\
My work on the Android application involves:
\begin{itemize}
\item collaborating with other developers, test analysts, business analysts and business people
\item using a bug tracker to find bugs relating to the application
\item using version control systems to get the latest source code from a centralised server, merging
changes into the main and release branches
\item fixing internal and customer visible bugs
\item implementing features designed by the business analyst on the team
\item having my work reviewed and quality tested
\item highly experienced using the following tools for android development:
	\begin{itemize}
	\item mockito - java testing framework
	\item gradle - build automation tool
	\item guice - lightweight dependency injection
	\item jenkins - continuous integration server
	\item otto bus - android specific event bus
	\item git - source control management
	\end{itemize}
\end{itemize}}
\cventry{2013--2014}{Student Intern}{Vista Entertainment Solutions}{Auckland}{}{
\begin{itemize}%
\item My role involved working on bugs in the cinema management software. The technologies used in that
area include WinForms, Visual Basic, C\#, .NET, and Microsoft SQL Server.
\item Some of my work included web development where I updated the user interface on a legacy website
that served pages using XSLT. Other technologies used were JavaScript, SASS, HTML, and ASP.NET.
\end{itemize}}

\section{Projects}
\cvitem{Java}{Developed a source to source compiler using JavaCC that took Java
8 style lambda expressions and generated valid Java 7 anonymous class
instantiations. The compiler also took care to check the semantics of how the
lambda expression were used and reported useful errors if it was used
incorrectly. Link to project:
\url{https://github.com/kurtmc/LambdaSourceToSource}}
\cvitem{C}{Developed a kernel module that can be dynamically loaded into the Linux kernel during run time.  It uses a
character device to take in a phrase as a string and return it reversed, i.e. ``Hello World'' becomes ``World Hello''.
Link to project: \url{https://github.com/kurtmc/reverse_module}}
\cvitem{C}{Developed a simple filesystem for the FUSE interface on linux using
C. It's simple filesystem that reads and writes to files in memory. Link to
project: \url{https://github.com/kurtmc/hello-world-fuse}}
\cvitem{Bash}{Developed a file renaming command line tool in bash. The tool detects a certain pattern in a group of
filenames and applies a new pattern to rename the files for improved readability. Link to project:
\url{https://github.com/kurtmc/renamer}}

\section{Favourite Linux tools}
\begin{tabular}{ll}
Text editor & vim \\
Web server  & nginx \\
Init system & systemd
\end{tabular}
\quad
\quad
\quad
\quad
\begin{tabular}{ll}
Desktop environment & GNOME \\
Package manager     & dnf \\
Distro              & fedora
\end{tabular}
\section{Interests}
\cvitem{Hacking the Linux kernel}{Checkout my kernel module on \href{https://github.com/kurtmc/reverse_module}{github}}
\cvitem{Maintaining Linux servers}{I maintain an 8TB raid 6 file server at home}
\cvitem{Community involvement}{I volunteered at the Auckland linux.conf.au in 2015, helping to setup the conference venue and pickup speakers from the airport}

\section{References available upon request}

\clearpage

%-----       letter       ---------------------------------------------------------

% Bunch of commands so that I can swap out different company names and people
\newcommand{\companyName}{<company name>}
\newcommand{\companyAddress}{Auckland\\New Zealand}
\newcommand{\roleName}{<role name>}
\newcommand{\recipientName}{Sir / Madam}

% recipient data
\recipient{\companyName{}}{\companyAddress{}}
\date{\today}
\opening{Dear \recipientName{},}
\closing{Yours faithfully,}
\enclosure[Attached]{curriculum vit\ae{}}
\makelettertitle

This application is for \roleName{} advertised by \companyName{}.

I have completed a four year Software Engineering degree at the University of
Auckland. I think that I can be extremely valuable to your company as I have a
great deal of real world experience as well as contemporary knowledge gained
through my university education.

I looking forward to meeting with you in the future to discuss this further.

\makeletterclosing

\end{document}
